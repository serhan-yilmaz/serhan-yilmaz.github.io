% Options for packages loaded elsewhere
\PassOptionsToPackage{unicode}{hyperref}
\PassOptionsToPackage{hyphens}{url}
%
\documentclass[
]{book}
\usepackage{amsmath,amssymb}
\usepackage{lmodern}
\usepackage{iftex}
\ifPDFTeX
  \usepackage[T1]{fontenc}
  \usepackage[utf8]{inputenc}
  \usepackage{textcomp} % provide euro and other symbols
\else % if luatex or xetex
  \usepackage{unicode-math}
  \defaultfontfeatures{Scale=MatchLowercase}
  \defaultfontfeatures[\rmfamily]{Ligatures=TeX,Scale=1}
\fi
% Use upquote if available, for straight quotes in verbatim environments
\IfFileExists{upquote.sty}{\usepackage{upquote}}{}
\IfFileExists{microtype.sty}{% use microtype if available
  \usepackage[]{microtype}
  \UseMicrotypeSet[protrusion]{basicmath} % disable protrusion for tt fonts
}{}
\makeatletter
\@ifundefined{KOMAClassName}{% if non-KOMA class
  \IfFileExists{parskip.sty}{%
    \usepackage{parskip}
  }{% else
    \setlength{\parindent}{0pt}
    \setlength{\parskip}{6pt plus 2pt minus 1pt}}
}{% if KOMA class
  \KOMAoptions{parskip=half}}
\makeatother
\usepackage{xcolor}
\usepackage{longtable,booktabs,array}
\usepackage{calc} % for calculating minipage widths
% Correct order of tables after \paragraph or \subparagraph
\usepackage{etoolbox}
\makeatletter
\patchcmd\longtable{\par}{\if@noskipsec\mbox{}\fi\par}{}{}
\makeatother
% Allow footnotes in longtable head/foot
\IfFileExists{footnotehyper.sty}{\usepackage{footnotehyper}}{\usepackage{footnote}}
\makesavenoteenv{longtable}
\usepackage{graphicx}
\makeatletter
\def\maxwidth{\ifdim\Gin@nat@width>\linewidth\linewidth\else\Gin@nat@width\fi}
\def\maxheight{\ifdim\Gin@nat@height>\textheight\textheight\else\Gin@nat@height\fi}
\makeatother
% Scale images if necessary, so that they will not overflow the page
% margins by default, and it is still possible to overwrite the defaults
% using explicit options in \includegraphics[width, height, ...]{}
\setkeys{Gin}{width=\maxwidth,height=\maxheight,keepaspectratio}
% Set default figure placement to htbp
\makeatletter
\def\fps@figure{htbp}
\makeatother
\setlength{\emergencystretch}{3em} % prevent overfull lines
\providecommand{\tightlist}{%
  \setlength{\itemsep}{0pt}\setlength{\parskip}{0pt}}
\setcounter{secnumdepth}{5}
\usepackage{booktabs}
\usepackage{amsthm}
\makeatletter
\def\thm@space@setup{%
  \thm@preskip=8pt plus 2pt minus 4pt
  \thm@postskip=\thm@preskip
}
\makeatother
\ifLuaTeX
  \usepackage{selnolig}  % disable illegal ligatures
\fi
\usepackage[]{natbib}
\bibliographystyle{apalike}
\IfFileExists{bookmark.sty}{\usepackage{bookmark}}{\usepackage{hyperref}}
\IfFileExists{xurl.sty}{\usepackage{xurl}}{} % add URL line breaks if available
\urlstyle{same} % disable monospaced font for URLs
\hypersetup{
  pdftitle={Compendium of Life Principles: How to live a satisfying life and form meaningful relationships?},
  pdfauthor={Serhan Yilmaz},
  hidelinks,
  pdfcreator={LaTeX via pandoc}}

\title{Compendium of Life Principles: How to live a satisfying life and form meaningful relationships?}
\author{Serhan Yilmaz}
\date{}

\begin{document}
\maketitle

{
\setcounter{tocdepth}{1}
\tableofcontents
}
\hypertarget{preface}{%
\chapter{Preface}\label{preface}}

The goal of this writing is to summarize a set of rules that I believe to be important and beneficial to follow while making personal decisions. Most are obvious principles, but they are sometimes overlooked. So, I believe it is helpful to remember and repeat what's important.

To explain my view with an analogy: Principles serve as a compass that, in a dark tunnel where the road ahead is unclear, can help find the right way out, without getting lost. So, I say follow the compass when in doubt.

Overall, this is meant to be a practical guide, aiming for what is best personally and ``what can help'' rather than prescribing morals on ``what should''. Since I believe actions are ultimately what counts, I expressed the principles in the form of actionable terms, with pointers for further concepts to learn.

With these in mind, the first principle and a warning to follow:

\hypertarget{principle-of-advice}{%
\section{Principle of Advice}\label{principle-of-advice}}

\begin{itemize}
\tightlist
\item
  Take all advice with a grain of salt, think critically and assess limitations.
\item
  Realize there is no single answer, multiple paths can reach the same destination.
\item
  But, don't ignore what is obvious. If everywhere smells shit, look under your shoe.
\item
  Most importantly, live and be true to yourself.
\end{itemize}

\hypertarget{relationships}{%
\chapter{Relationships}\label{relationships}}

\hypertarget{principle-of-self-value}{%
\section{Principle of Self-Value}\label{principle-of-self-value}}

\begin{itemize}
\tightlist
\item
  Value yourself first.
\item
  Stand up for yourself, don't let others mistreat you.
\item
  Acknowledge that you are not responsible for other people's happiness.
\item
  Take responsibility for yourself and your happiness.
\item
  Limit altruism, do not sacrifice yourself for others.
\item
  Not to say be selfish, but don't be self-less either. Have an identity.
\item
  Don't let your value be defined by your value to others. Remember your value to yourself.
\end{itemize}

\hypertarget{principle-of-nonmaleficence}{%
\section{Principle of Nonmaleficence}\label{principle-of-nonmaleficence}}

\begin{itemize}
\tightlist
\item
  Do not harm others if they don't harm you or anyone else.
\item
  Do not tolerate, trust or accept those who would use and harm others for selfish reasons.
\item
  If retaliation is warranted, make sure magnitude is proportional to, and not exceed the intended harm or actual harm (whichever is greater).
\item
  (Try to) Protect those who are in need, who are harmed by others for selfish reasons. Particularly those who lack the power to protect themselves.

  \begin{itemize}
  \tightlist
  \item
    Only if you can. Don't be a hero. See principle of self-value, you are not responsible for other people's happiness.
  \item
    But remember the poem: \href{https://en.wikipedia.org/wiki/First_they_came_...}{``First they came for\ldots{}''}
  \end{itemize}
\end{itemize}

\hypertarget{principle-of-loyalty}{%
\section{Principle of Loyalty}\label{principle-of-loyalty}}

\begin{itemize}
\tightlist
\item
  Do not betray those who are loyal to you.
\item
  Be loyal to those you trust, and aim for mutual benefit.
\item
  Choose your allies well. Go for people who won't betray you but will lift you up.
\item
  In return, provide support and encouragement to your allies. Lift them up.
\item
  Obey the rule of reciprocity in your interactions, for as long as you wish to sustain a connection.
\end{itemize}

\hypertarget{principle-of-reciprocity}{%
\section{Principle of Reciprocity}\label{principle-of-reciprocity}}

\begin{itemize}
\tightlist
\item
  Follow \href{https://en.wikipedia.org/wiki/Reciprocity_(social_psychology)}{the rules of reciprocity}, to form or continue a connection:

  \begin{itemize}
  \tightlist
  \item
    Key loop: Give what you take \textless=\textgreater{} Receive what you give.\\
  \item
    Reciprocate when you receive a favor, if feasible.
  \item
    Don't be transactional. Give without an expectation to receive. Don't give when you can't afford not to receive.
  \item
    If your favor is not reciprocated, don't blame or demand. But also, don't repeat if the favor is costly to you.

    \begin{itemize}
    \tightlist
    \item
      If this happens, take it as a signal that the other party isn't interested in connecting with you. So, stop the loop.
    \item
      Similarly, don't reciprocate if you would like to stop the loop. It doesn't have to last forever.
    \end{itemize}
  \item
    While reciprocating, aim for a favor that is equal or slightly more in magnitude than what you received.
  \item
    Start small to initiate the loop, and grow your way up.
  \item
    Ignore small variations and violations as trust builds up.
  \end{itemize}
\item
  Understand \href{https://en.wikipedia.org/wiki/Reciprocal_altruism}{reciprocal altruism}. That's the nature's way of establishing cooperation. Trust it.
\item
  Clarify beforehand if you will expect something back from a favor.

  \begin{itemize}
  \tightlist
  \item
    That's a transaction. Negotiate and come to an agreement beforehand. Verbally if it is someone you trust, or by written contract.
  \item
    Do not hold any previous favors you've done as leverage, particularly unsolicited. That would be manipulation. Deals are negotiated beforehand.
  \end{itemize}
\item
  Exercise caution and assess motivation behind any favors.

  \begin{itemize}
  \tightlist
  \item
    Check out \protect\hyperlink{reciprocal-relations-and-goodwill}{the notes on reciprocal relations and goodwill} for further reading.
  \end{itemize}
\item
  When in doubt, match the other person's efforts.
\end{itemize}

\hypertarget{principle-of-cooperation}{%
\section{Principle of Cooperation}\label{principle-of-cooperation}}

\begin{itemize}
\tightlist
\item
  Aim to maximize mutual benefit.

  \begin{itemize}
  \tightlist
  \item
    Key is to increase total gain by working together.
  \item
    Don't ask for favors that provide little benefit to you but impose a big burden on the other person. Likewise, don't accept requests that comes with great cost to you but only a small benefit to the other person.
  \end{itemize}
\item
  Engage in productive exchanges. In many cases, it is easy to contribute to another person's happiness with minimal cost to yourself.

  \begin{itemize}
  \tightlist
  \item
    Small acts of kindness can serve this purpose.
  \item
    Take initiative sometimes. It can pay off.
  \end{itemize}
\item
  Defend your interests first. Cooperate only when the interests align and the pay-off will be beneficial to both parties, in short-term or long-term.

  \begin{itemize}
  \tightlist
  \item
    Not always for foreseeable gains. Give leeway when there is little cost to you.
  \item
    Do good for the sake of doing good, sometimes.
  \end{itemize}
\end{itemize}

\hypertarget{principle-of-respect}{%
\section{Principle of Respect}\label{principle-of-respect}}

\begin{itemize}
\tightlist
\item
  Do not think or claim you are more important than others.
  Everyone is important to themselves from their perspective.

  \begin{itemize}
  \tightlist
  \item
    Be aware that your importance to others will generally be proportional to what you can contribute to their lives. Wish to be more important? Contribute more.
  \item
    Or better yet, have some self-esteem and be important to yourself first. Outside validation will never be sufficient by itself anyway.
  \end{itemize}
\item
  Do not judge anyone. But, always assess.

  \begin{itemize}
  \tightlist
  \item
    Assess their potential to be a trustworthy and valuable ally.
  \item
    Also, evaluate any risks involved with someone. Don't wait for harm to come, take any necessary precautions. Disconnect if necessary. Remember you are responsible for your well-being.
  \end{itemize}
\item
  Avoid stereotypes. Do not make broad assumptions.

  \begin{itemize}
  \tightlist
  \item
    The world and people are complex. Consider everyone as an individual. Realize that group traits rarely explain individual differences adequately.
  \end{itemize}
\item
  Always be kind, unless otherwise warranted. Don't be an asshole without reason.
\item
  Admit your mistakes. Try to compensate any damage to others, if feasible.

  \begin{itemize}
  \tightlist
  \item
    Similarly, respect those who admit their mistakes and offer them a chance at redemption. People that try to hide their mistakes wouldn't be great allies.
  \end{itemize}
\item
  Do not believe you are so clever and can manipulate or cheat your way easily. People aren't stupid. The vibe you give will reflect your intentions.

  \begin{itemize}
  \tightlist
  \item
    In many cases, selfish deeds will come back to bite you, sooner or later.
  \item
    Instead, be sincere and approach people with goodwill. Follow the principle of authenticity to form close relationships.
  \end{itemize}
\end{itemize}

\hypertarget{principle-of-authenticity}{%
\section{Principle of Authenticity}\label{principle-of-authenticity}}

\begin{itemize}
\tightlist
\item
  Be honest with yourself, always.
\item
  Do what you say. Say what you do. Follow through your promises.
\item
  Don't lie or mislead others without good reason, especially for personal benefit.
\item
  Be open and honest in your interactions, with people you trust.
\item
  Have people in your life you can trust, where you can be yourself without judgement. Prioritize and make effort to form such close relationships.
\item
  \href{https://en.wikipedia.org/wiki/Golden_Rule}{Treat others how you want to be treated yourself}.

  \begin{itemize}
  \tightlist
  \item
    Better yet, treat them how they want to be treated. When feasible.
  \end{itemize}
\item
  Be open with your feelings and aspirations. First, with yourself. Then, with others you trust. This is important.
\item
  Express your thoughts and feelings clearly. Especially when you have a concern. Don't play it off as a joke, nor let someone else dismiss it as a joke.

  \begin{itemize}
  \tightlist
  \item
    Learn to recognize \href{https://en.wikipedia.org/wiki/Passive-aggressive_behavior}{passive-aggressive behaviors}. Prefer \href{https://en.wikipedia.org/wiki/Assertiveness\#Communication}{assertive communication} over aggressive or indirect means, unless warranted otherwise.
  \end{itemize}
\item
  Know your limits. Be assertive in establishing your boundaries. Don't shy away from conflict if violated. But, pick your fights. Goal is to defend. Avoid unnecessary fights, especially if this is not an repeated interaction. Compromise as necessary.

  \begin{itemize}
  \tightlist
  \item
    In trivial disputes, let the person who cares more have their way, if mutually beneficial cooperation is feasible. But, do not agree what you believe to be untrue. Be true to yourself and defend what's important.
  \item
    In conflicts involving repeated interactions, aim for a compromise when possible, unless there is a breach of trust. Threats, manipulation, dishonesty and alike would be immediate violations of trust. Deal with aggression as appropriate.
  \end{itemize}
\end{itemize}

\hypertarget{principle-of-humility}{%
\section{Principle of Humility}\label{principle-of-humility}}

\begin{itemize}
\tightlist
\item
  Accept people as they are and acknowledge that you can't change them.

  \begin{itemize}
  \tightlist
  \item
    You do not have such power, only they do.
  \end{itemize}
\item
  Be realistic. Admit your weaknesses. But, don't be too hard on yourself.

  \begin{itemize}
  \tightlist
  \item
    Accept what you cannot change. Prioritize what you can improve.
  \end{itemize}
\item
  Accept when you are wrong. It's okay to be wrong. Can't always be right.

  \begin{itemize}
  \tightlist
  \item
    Do not attribute your self-worth to being right. So what, if you are wrong?
  \end{itemize}
\item
  Remember that you need other people. Can't do everything by yourself.

  \begin{itemize}
  \tightlist
  \item
    Can't survive alone, nor live a long, fulfilling life without others.
  \item
    Ask for help, if needed.
  \end{itemize}
\item
  Acknowledge that your thoughts or feelings do not always reflect reality accurately.

  \begin{itemize}
  \tightlist
  \item
    Admit that you can't know or predict everything. Uncertainty is everywhere in life. Avoid absolute judgements.
  \item
    Check out \href{https://en.wikipedia.org/wiki/Cognitive_distortion}{cognitive distortions} and \protect\hyperlink{cognitive-distortions}{the related notes below}.
  \end{itemize}
\item
  Realize that your power is limited and you cannot solve every problem.

  \begin{itemize}
  \tightlist
  \item
    Do not get lost in group struggles and problems of the world.
  \item
    Focus on what you can solve and control. Primarily your actions. Followed by your neighborhood. Check out \href{https://en.wikipedia.org/wiki/Stoicism}{stoic philosophy}.
  \end{itemize}
\end{itemize}

\hypertarget{principle-of-emotionality}{%
\section{Principle of Emotionality}\label{principle-of-emotionality}}

\begin{itemize}
\tightlist
\item
  Realize that humans behave emotionally as much as they do logically.
  Learn the necessary skills to navigate both landscapes.
\item
  This is multi-faceted. Consider the following:

  \begin{itemize}
  \tightlist
  \item
    \href{https://en.wikipedia.org/wiki/Self-awareness}{Emotional awareness} for starters. Learn to name your feelings and be aware of them. Make use of \href{https://commons.wikimedia.org/wiki/File:The_Feeling_Wheel.png}{feeling wheel}.
  \item
    \href{https://en.wikipedia.org/wiki/Emotional_self-regulation}{Emotional regulation}. Learn to manage your emotions without being damaging.
  \item
    \href{https://en.wikipedia.org/wiki/Emotional_expression}{Emotional expression}. Learn to convey your emotions effectively. Acknowledge that it's a subjective experience. Prefer \href{https://en.wikipedia.org/wiki/I-message}{``I'' language} when possible. Don't forget to express positive emotions like love and gratitude.
  \item
    \href{https://en.wikipedia.org/wiki/Empathy}{Empathy}. Learn to be socially aware and recognize emotions of others.

    \begin{itemize}
    \tightlist
    \item
      Level one, place yourself in other person's position and assess how you would feel.
    \item
      Level two, assess how they would feel based on their beliefs and perception that are different than yours.
    \item
      This skill usually brings compassion as well.
    \end{itemize}
  \item
    \href{https://en.wikipedia.org/wiki/Emotional_validation}{Emotional validation}. Learn to convey that you understand and accept other person's emotions. Relate to the situation if you can, ``I would have felt the same in that situation''. Realize that you do not have to condone actions.
  \end{itemize}
\item
  A caveat: Acknowledge, but don't believe your feelings blindly. They are quite real themselves, but not necessarily what they imply.

  \begin{itemize}
  \tightlist
  \item
    For example, suppose you feel guilty. This doesn't necessarily mean you are guilty or made a mistake. Similarly, if you feel anxious, doesn't necessarily mean you are in danger.
  \item
    Negative emotions are warnings, but they can be false alarms. Use logic to discern. Check out \href{https://en.wikipedia.org/wiki/Emotional_reasoning}{emotional\_reasoning} for more information.
  \end{itemize}
\end{itemize}

\hypertarget{principle-of-attachment}{%
\section{Principle of Attachment}\label{principle-of-attachment}}

\begin{itemize}
\tightlist
\item
  Be capable of accepting loss. Don't try to avoid the inevitable.

  \begin{itemize}
  \tightlist
  \item
    Understand \href{https://en.wikipedia.org/wiki/Sunk_cost\#Fallacy_effect}{sunk cost fallacy}, and try not to fall into it.
  \end{itemize}
\item
  Don't be afraid to attach, it doesn't have to be permanent.
\item
  Walk away, if a relationship becomes consistently harmful or otherwise not worth it in the long run.
\item
  Check out \href{https://en.m.wikipedia.org/wiki/Attachment_in_adults\#Styles}{attachment styles}.
\end{itemize}

\hypertarget{principle-of-intimacy}{%
\section{Principle of Intimacy}\label{principle-of-intimacy}}

\begin{itemize}
\tightlist
\item
  Aim for consensual, mutual pleasure in sexual relations.
\item
  Be safe first, then be open and playful.
\item
  Don't underestimate the effect of mentality, emotionality, and power dynamics.
\item
  Don't assume porn and erotica reflect reality.
\item
  Check out \href{https://en.wikipedia.org/wiki/The_Five_Love_Languages}{love languages}. Speak the same language.

  \begin{enumerate}
  \def\labelenumi{\arabic{enumi})}
  \tightlist
  \item
    Words of affirmation and compliments, 2) Quality time, 3) Acts of service, 4) Physical touch, and 5) Gifts.
  \end{enumerate}
\end{itemize}

\hypertarget{key-takeaways}{%
\section{Key Takeaways}\label{key-takeaways}}

Overall, relationships can be complex and nuanced, as they require a delicate balance. There is indeed a lot to balance. While these apply to more than relationships, heed the following in your interactions:

\hypertarget{principle-of-balance}{%
\subsection{Principle of Balance}\label{principle-of-balance}}

\begin{itemize}
\tightlist
\item
  Balance being selfless and being selfish.
\item
  Balance giving and taking.
\item
  Balance emotionality with logic.
\item
  Balance serious with humor.
\item
  Balance being carefree with respect.
\item
  Balance asserting boundaries with compromise.
\item
  Balance optimism and trust with caution.
\item
  Balance future concern with living in the moment.
\item
  Balance hope with effort.
\item
  Balance rules and responsibilities with enjoyment.
\item
  Balance staying in comfort zone with taking initiative.
\item
  Balance foreseeable with exploration of the unknown.
\item
  Balance being dependable without falling into predictable.
\item
  Finally, balance self-interest with a genuine interest in others.
\item
  Most importantly, balance valuing others with valuing yourself.
\end{itemize}

\hypertarget{notes}{%
\section{Notes}\label{notes}}

\hypertarget{reciprocal-relations-and-goodwill}{%
\subsection{Reciprocal Relations and Goodwill}\label{reciprocal-relations-and-goodwill}}

\begin{itemize}
\tightlist
\item
  Ally yourself with others who will engage in reciprocal behaviour in good will.
\item
  Prioritize forming mutually beneficial connections. Engage in reciprocal behavior and do favors because you care.

  \begin{itemize}
  \tightlist
  \item
    Not because you have something in mind that you would like to receive.
  \end{itemize}
\item
  Accept favors proportional to what you are willing to give.
\item
  Exercise caution. If someone haven't earned your trust yet, assess the motivation behind any favors you receive.

  \begin{itemize}
  \tightlist
  \item
    Is the intention to connect, or is it to leave you indebted before making a request? If it seems like the latter, disengage. It's not worth it.
  \item
    Is the favor big? Refuse, or proceed with extreme caution. Disengage if there are any strings attached, however small.
  \item
    Does the person has a habit of gossip? Refuse, unless the favor is minor or truly needed.

    \begin{itemize}
    \tightlist
    \item
      Instead, start the reciprocal loop by offering a favor yourself, if you would like to connect.\\
    \item
      Disengage, if they don't accept any favors and are only interested in offering on their terms. Their interest may lie in gaining power over you rather than to connect.
    \end{itemize}
  \end{itemize}
\item
  Acknowledge that you are not obligated to fulfill a request in return for a previous favor, particularly from someone you don't trust yet.

  \begin{itemize}
  \tightlist
  \item
    You can always reciprocate later and make up for it in a way of your choosing. Doesn't violate loyalty.
  \item
    A bad actor may not be happy with this. Not a big issue for someone whose aim is to connect.
  \item
    However, lean towards acceptance if it's going to be a huge help to the requestor while not being a substantial cost to you. Assess risks and potential. See it as an opportunity to connect, if desirable.
  \end{itemize}
\item
  Communicate values clearly. Express appreciation appropriately.

  \begin{itemize}
  \tightlist
  \item
    Don't pretend something is a huge help while it is not. Likewise, do not understate if something was truly helpful.
  \item
    Do not negotiate favors. Only clarify the benefits and costs to you.

    \begin{itemize}
    \tightlist
    \item
      Negotiate transactions, not favors.
    \end{itemize}
  \item
    Never criticize a behavior that you would like to see repeated. Instead, validate and give positive feedback by expressing appreciation.\\
  \end{itemize}
\item
  Address the person themself in your communications, particularly in conflicts.

  \begin{itemize}
  \tightlist
  \item
    Do not gossip, nor bring other people in the middle of your interactions.
  \item
    Likewise, do not trust or ally with someone who engage in such behavior, unless necessary.
  \end{itemize}
\item
  Try to leave good impressions, it makes a difference.

  \begin{itemize}
  \tightlist
  \item
    However, do not aim to impress people deliberately. That can backfire easily.
  \end{itemize}
\item
  Understand that \href{https://en.wikipedia.org/wiki/Costly_signaling_theory_in_evolutionary_psychology}{costly actions are effective signals}.

  \begin{itemize}
  \tightlist
  \item
    They are effective because: They are difficult to fake. For example, carrying expensive jewelry works as a signal of wealth because not everyone can accept that cost.
  \item
    This applies to demonstrating goodwill as well. Time, effort and the resulting history can function as big costs for this purpose. Similarly, the ability to handle rejections or losses gracefully can be an effective indicator of goodwill.
  \end{itemize}
\end{itemize}

\hypertarget{cognitive-distortions}{%
\subsection{Cognitive Distortions}\label{cognitive-distortions}}

\begin{itemize}
\tightlist
\item
  Understand \href{https://en.wikipedia.org/wiki/Black_swan_theory}{black swan counter-example} and recognize that unpredictable, singular events can occur. Avoid \href{https://en.wikipedia.org/wiki/Splitting_(psychology)}{black and white thinking} and do not make generalized statements that are absolute.

  \begin{itemize}
  \tightlist
  \item
    ``All swans are white'' is an absolute and unprovable statement. But, one counter-example is enough to disprove it.

    \begin{itemize}
    \tightlist
    \item
      Thus, such statements are unlikely to be correct. Exceptions happen.
    \end{itemize}
  \item
    \emph{Over-generalization}: Making sweeping conclusions based on a few events.

    \begin{itemize}
    \tightlist
    \item
      Example: All woman are liars, and therefore, cannot be trusted.
    \end{itemize}
  \item
    \emph{All or nothing thinking}: Dealing in extremes, either absolutely good or bad.

    \begin{itemize}
    \tightlist
    \item
      Example: I was only able to get third place in the race. I am a failure.
    \item
      Another example: I was on a diet, but ate chips after dinner. I fucked up. I've ruined my diet now. I am a complete failure.
    \item
      In reality, magnitudes matter. There is an in-between.
    \item
      Think probabilistic, not deterministic. There is uncertainty.
    \end{itemize}
  \end{itemize}
\item
  Do not \href{https://en.wikipedia.org/wiki/Jumping_to_conclusions}{jump to conclusions}. Realize that you can't know or predict everything.

  \begin{itemize}
  \tightlist
  \item
    Acknowledge that uncertainty is everywhere in life. Avoid absolute judgements.
  \item
    \emph{Mind reading}: Assuming what someone must be thinking or feeling, beyond reason. Avoid making big conclusions based on minor behaviors. Behaviors only go so far in understanding thoughts and intentions.

    \begin{itemize}
    \tightlist
    \item
      Example: I know you don't like me. How so? You didn't invite me to lunch.
    \end{itemize}
  \item
    \emph{Fortune-telling}: Predicting outcomes in absolute terms.

    \begin{itemize}
    \tightlist
    \item
      Example: A depressed person believing they will never improve and nothing will ever be resolved for their whole life.
    \item
      Another example: I'll never find a job. I'll be unemployed forever.
    \end{itemize}
  \item
    \emph{Labeling}: Overly-reductive description of other people's characteristics.

    \begin{itemize}
    \tightlist
    \item
      Example: He didn't call me. He must be a narcissist!
    \end{itemize}
  \end{itemize}
\item
  Be mindful of behaviors that turn into \href{https://en.wikipedia.org/wiki/Self-fulfilling_prophecy}{self-fulfilling prophecies}.

  \begin{itemize}
  \tightlist
  \item
    ``You create your own reality'' phrase has a point. Beliefs heavily influence perception and behaviors. They can decisively affect outcomes, particularly when it comes to relationships.

    \begin{itemize}
    \tightlist
    \item
      For example, rejection expectations can lead to \href{https://doi.org/10.1037/0022-3514.75.2.545}{behaviors that elicit rejection from others}.
    \item
      They also shape your subjective experiences to a large degree. \href{https://en.wikipedia.org/wiki/Placebo\#Effects}{Placebo effect} comes to mind.
    \end{itemize}
  \item
    Optimism and having a positive outlook help in human relations. The vibe you give to others is shaped by your perspective and mood.

    \begin{itemize}
    \tightlist
    \item
      Note that, faking doesn't necessarily help much. Believing does.
    \end{itemize}
  \item
    Believe that future is not fixed and can be changed if not desirable. Exert will and take action. Focus on things that can help, one step at a time.

    \begin{itemize}
    \tightlist
    \item
      You may be unsure whether you will be able to achieve your goals by trying. But, one thing is near-certain: You won't succeed if you never try. Assess pros and cons, then make your move.
    \end{itemize}
  \item
    Examples of defeatist attitude that turn into self-fulfilling prophecies:

    \begin{itemize}
    \tightlist
    \item
      I won't study to the exam. I am going to fail anyway.
    \item
      I can't make any friends. People don't like me. I better stay away.
    \item
      I'm terrible at math. I'll never understand this lesson. No use listening.
    \item
      I won't even try asking out. They'll definitely reject me. I'll be alone forever.
    \item
      No one can be trusted. Must be vigilant. Better them than me.\\
    \end{itemize}
  \end{itemize}
\item
  \href{https://en.wikipedia.org/wiki/Emotional_reasoning}{Emotional Reasoning}: Believing emotions as facts, independent of any empirical evidence.

  \begin{itemize}
  \tightlist
  \item
    Example: I feel jealous, which means my partner must be unfaithful. I wouldn't feel jealous if my partner were faithful.
  \item
    Another example: I feel overwhelmed. It's impossible to even start.
  \end{itemize}
\end{itemize}

\hypertarget{happiness}{%
\chapter{Happiness}\label{happiness}}

\hypertarget{principle-of-effort}{%
\section{Principle of Effort}\label{principle-of-effort}}

\begin{itemize}
\tightlist
\item
  Take responsibility for your happiness and well-being.

  \begin{itemize}
  \tightlist
  \item
    As an adult, it is your responsibility. Take reins, don't take the role of a victim.\\
  \end{itemize}
\item
  Focus on what you can control.

  \begin{itemize}
  \tightlist
  \item
    Focus on what's ahead and what you can change. Don't get stuck living in the past. Leave out what you cannot control. Make peace with them. No one is omnipotent.
  \item
    Don't take responsibility for events out of your control. Take responsibility for what you can influence. This is the essence of stoic philosophy.
  \end{itemize}
\item
  Take action. Do what you need to do.

  \begin{itemize}
  \tightlist
  \item
    Fulfill your needs first. Ensure your safety.
  \item
    Take care of your health and physical well-being. Also, mental health.
  \item
    Build strong interpersonal relationships that feel meaningful.
  \item
    Ask: What would make you proud of yourself?
  \end{itemize}
\item
  Check out \href{https://en.wikipedia.org/wiki/Maslow\textquotesingle{}s_hierarchy_of_needs}{Maslow's hierarchy of needs}.

  \begin{itemize}
  \tightlist
  \item
    Physiological needs first. Food, sleep, shelter and so on.
  \item
    Safety needs afterwards. Personal, emotional and financial. Also, health.
  \item
    Love and social needs comes next. Trust, acceptance, intimacy, sex. Friends, family, community. A sense of belonging.
  \item
    Esteem needs later. Self-respect, confidence, competence. Being proud of yourself.
  \item
    What comes afterwards? Living true to yourself. How does it happen? Aligning your beliefs and actions. Doing what you believe is right. Having a sense of purpose and meaning.
  \end{itemize}
\end{itemize}

\hypertarget{principle-of-serenity}{%
\section{Principle of Serenity}\label{principle-of-serenity}}

\begin{itemize}
\tightlist
\item
  Learn to be content with what you have.

  \begin{itemize}
  \tightlist
  \item
    Limit wants. Wants have no end if left unbounded, since it is human desire to always want more. Aim to understand why do you want what you want. What do you truly desire?
  \item
    This doesn't mean don't be ambitious. Quite the contrary. Exercise actions you deem appropriate, but leave out the rest. Don't mull over what could or should have happened.
  \item
    Don't compare yourself with others. Comparison is a thief of joy.
  \end{itemize}
\item
  Be cautiously optimistic.

  \begin{itemize}
  \tightlist
  \item
    In essence, hope for the best and prepare for the worst.
  \item
    Be realistic, don't neglect to take precautions. But, stay optimistic afterwards. Believe that it is going to be okay, one way or another. Don't worry too much, especially for things out of your control.
  \item
    Appreciate the good in your life. See the glass \href{https://en.wikipedia.org/wiki/Is_the_glass_half_empty_or_half_full\%3F}{half full}.\\
  \end{itemize}
\item
  Don't take yourself or life so seriously.

  \begin{itemize}
  \tightlist
  \item
    A little humor goes a long way. Life is too short to worry too much.
  \item
    Stop to smell the roses along the way. Appreciate beauty.
  \item
    Don't let work define you. View it as a part, a means for a purpose.
  \end{itemize}
\end{itemize}

\hypertarget{principle-of-purpose}{%
\section{Principle of Purpose}\label{principle-of-purpose}}

\begin{itemize}
\tightlist
\item
  Focus on the process, not outcome.

  \begin{itemize}
  \tightlist
  \item
    Life is not a race, don't rush till the end. Grand finale is already determined from the beginning: Your life will come to an end.
  \item
    Don't forget the future and the long term. But also, don't forget there is an end. Do the things you want to do while you can.
  \item
    Enjoy the process. Be open to trying new things and explore.
  \end{itemize}
\item
  Consider right actions, rather than right outcomes.

  \begin{itemize}
  \tightlist
  \item
    \href{https://en.wikipedia.org/wiki/Buddhist_philosophy}{Buddhist philosophy:} The source of frustration is not losing. It is the desire to win, the desire to reach a particular outcome. \href{https://existentialcomics.com/comic/102}{Related comic}.
  \item
    Doing and expecting are different. Do, but don't expect.
  \item
    Engage in actions because they are right, given the available knowledge. Do not regret a contrary or undesirable outcome of an action that seemed right at the time. See it as a learning experience, it is part of the process.\\
  \item
    Be content in knowing that you did what you are supposed to do, regardless of outcome.
  \end{itemize}
\item
  Aim for a meaningful life, not for happiness.

  \begin{itemize}
  \tightlist
  \item
    Aiming for happiness typically leads to comfort and pleasure seeking behavior. Successful in the short run, but detrimental in the long run for happiness.
  \item
    Paradoxically, aiming for what feels meaningful leads to behaviour that reinforce happiness in the long run.

    \begin{itemize}
    \tightlist
    \item
      What feels meaningful is typically productive actions that involve other people, that are beneficial to others and the community. Engaging in such behavior typically results in positive feedback and improve personal satisfaction.
    \end{itemize}
  \end{itemize}
\item
  Deal with reality, and accept it. Describe and understand how things happen. Do not assert how things are supposed to happen.

  \begin{itemize}
  \tightlist
  \item
    Learn to distinguish fantasy from reality, opinion from fact, impressions from knowledge. Acknowledge subjectivity in life.
  \end{itemize}
\end{itemize}

\hypertarget{perspectives}{%
\section{Perspectives}\label{perspectives}}

\hypertarget{about-the-meaning-and-purpose-of-life}{%
\subsection{About the meaning and purpose of life}\label{about-the-meaning-and-purpose-of-life}}

My take: ``What's the meaning of life?'' is a backwards question, and the answer is nothing definite, because:

\begin{itemize}
\item
  Life doesn't exist because there is any meaning or purpose. But, meaning and purpose exists because there is life. That's where everything starts.

  \begin{itemize}
  \tightlist
  \item
    Our cognition and perspective is what attributes meaning to the external world. This makes our existence valuable to ourselves in a way that anything external can possibly hope not. I mean, if you lose that consciousness, what even remains there to discuss?
  \end{itemize}
\item
  Philosophies that defend similar ideas, and my interpretations:

  \begin{itemize}
  \tightlist
  \item
    \href{https://en.wikipedia.org/wiki/Existentialism}{\textbf{Existentialism:}} The life does not have a preordained meaning. Existence is what creates meaning, and that's what matters. So, live true to yourself and create your own purpose.
  \item
    \href{https://en.wikipedia.org/wiki/Nihilism}{\textbf{Nihilism:}} Life is devoid of any objective meaning, purpose, or intrinsic value. Individuals must either accept the inherent meaninglessness of existence or define their own meaning.
  \item
    \href{https://en.wikipedia.org/wiki/Absurdism}{\textbf{Absurdism:}} Humans inherently seek meaning and purpose in life, but the universe is indifferent and inherently meaningless. So, rebel against this absurd universe by determining your own purpose and values!
  \item
    \href{https://en.wikipedia.org/wiki/Taoism}{\textbf{Taoism:}} Do not impose external purposes. Embrace the natural flow of life. Focus on harmony and live in accordance with nature.
  \item
    \href{https://en.wikipedia.org/wiki/Buddhist_philosophy}{\textbf{Buddhism:}} Concept of \href{https://en.wikipedia.org/wiki/Anatt\%C4\%81}{Anatta (not-self)}. There is no permanent self. Life is continuous process of change. Thus, search for a fixed purpose or identity is a source of suffering. So, let go of any attachments or desires for specific purposes and focus on inner peace.
  \item
    \href{https://en.wikipedia.org/wiki/Stoicism}{\textbf{Stoicism:}} Happiness does not depend on external circumstances. It depends on your own virtue and rationality. So, focus on what's in your control and live virtuously in accordance with the nature. In other words: Do the right thing, based on what you know about the rational world, to find contentment and purpose in life.
  \end{itemize}
\end{itemize}

\hypertarget{about-wisdom-and-intelligence}{%
\subsection{About wisdom and intelligence}\label{about-wisdom-and-intelligence}}

Here is my perspective on the importance of wisdom in broader picture:

\begin{itemize}
\tightlist
\item
  \emph{Observation} is the process of learning, and collecting data.
\item
  \emph{Intelligence} is transforming data into knowledge.
\item
  \emph{Wisdom} is translating knowledge into appropriate actions, often in the face of uncertainty.
\item
  What remains after is executing these actions.
\end{itemize}

All steps are important in the end, since a chain is only as strong as its weakest link. That being said, to make sure that you truly understand a topic or information, aim to translate it into actions, answering:

\begin{itemize}
\tightlist
\item
  So what? What does this mean? What am I supposed to do with this information?
\end{itemize}

Without that, knowledge is just a dry description of reality, barely relevant.

\hypertarget{about-self-actualization---last-step-in-mazlovs-hierarchy-of-needs}{%
\subsection{About Self-Actualization - Last step in Mazlov's hierarchy of needs}\label{about-self-actualization---last-step-in-mazlovs-hierarchy-of-needs}}

According to Mazlov, \href{https://en.wikipedia.org/wiki/Self-actualization}{self-actualization} is the highest step where personal potential is fully realized after other bodily and ego needs are fulfilled. It is commonly interpreted as ``the full realization of one's potential'' or of one's ``true self'', and is associated with a person who is living creatively, autonomously, and spontaneously.

\textbf{My perspective}: I am not convinced about the importance of self-actualization in the form described by Mazlov. Or rather, about whether search for any external activities, creativity or ``passion'' is truly necessary. Nevertheless, I believe a sense of purpose is required to be content and to feel that life is worth living, whatever that purpose may be. This concept is called \href{https://en.wikipedia.org/wiki/Ikigai}{Ikigai} in Japanese culture, meaning `a reason for being'.

In my opinion, the answer lies in beliefs. The key to achieve this last step of fulfillment, after the other needs, is to behave in a way that is consistent with what you believe to be true, so that you can enjoy life and be proud of yourself.

In general, there are two ways this can happen:

\begin{itemize}
\tightlist
\item
  Believe what you do is right.
\item
  Do what you believe is right.
\end{itemize}

The answer will likely be a mixture of the two. Awareness and being honest with yourself is important for that purpose. Note that, a conscious thought is not necessarily an internal belief. Just like, saying something out loud doesn't necessarily make it true. So, this discussion is about internal beliefs, whether they are conscious or not.

\textbf{To summarize my opinion}: For someone to be content with themselves, their beliefs and actions need to be aligned so that they can be proud of themselves. Believing that you are important and worth being happy is vital. If you believe that everyone is important to themselves, thus worth being happy, this can make it easier for you to believe you are worthy as well. Being non-judgemental helps, though I don't consider this general belief to be necessary in all cases for everyone.

Some related concepts to this are \href{https://en.wikipedia.org/wiki/Cognitive_dissonance}{cognitive dissonance in psychology} and lowering expectations of future. See the discussions below for more information.

\hypertarget{about-cognitive-dissonance}{%
\subsection{About Cognitive Dissonance}\label{about-cognitive-dissonance}}

Cognitive Dissonance is a type of psychological stress that happens when the beliefs and actions, or rather the external world, are not aligned. Since this is a mentally taxing situation, the mind wants to resolve this situation in any way possible. Thus, a common response is \href{https://en.wikipedia.org/wiki/Rationalization_(psychology)}{rationalization}, or in other words: ``Believe what you do is right'' (the first option in the previous discussion).

In practice, rationalization can cause issues and can even be pathological, or otherwise not be beneficial. However, that is not my point here. My point is: This is a stressful situation. Alignment of the mind and external world is critical. This alignment is necessary to be content and to be at peace.

That's why I consider it a useful practice to go over the internal beliefs, questioning them or aligning them with the external events and the reality.

\hypertarget{about-lowering-expectations-and-living-in-the-moment}{%
\subsection{About lowering expectations and living in the moment}\label{about-lowering-expectations-and-living-in-the-moment}}

This concept is about: Having no or low expectations. More specifically, keeping your beliefs regarding the future minimal. This idea is particularly prominent in Buddhist and Stoic philosophies, as far as I know.

In general, the fewer and more flexible the beliefs are regarding the outer world and the future (while acknowledging any uncertainties), the easier it will be to align them with external events and be at peace. I believe this is the true importance of having an open mind. Thinking rigidly in terms of shoulds and musts, regarding how the world is supposed to function, is not beneficial in many circumstances.

Though of course, I'd also say: Do what needs to be done, to achieve fulfillment. This minimalism of low expectations is not an argument against ambitious actions. I just say: Keep your beliefs, regarding how the external world and other people should behave, simple and open-ended. Respond and behave in line with your internal beliefs, aiming to keep yourself and internal peace safe.

This idea is also closely related to \href{https://en.wikipedia.org/wiki/Carpe_diem}{Carpe diem (seize the day)} and living in the moment. Philosophies that teach similar ideas, and my interpretations:

\begin{itemize}
\tightlist
\item
  \href{https://en.wikipedia.org/wiki/Stoicism}{\textbf{Stoicism:}} Live in the present moment and focus on what is within your control.
\item
  \href{https://en.wikipedia.org/wiki/Buddhist_philosophy}{\textbf{Buddhism:}} Live and experience the present moment. Life is like a river, ever-changing. Nothing is permanent, so do not get overly attached.
\item
  \href{https://en.wikipedia.org/wiki/Taoism}{\textbf{Taoism:}} Concept of \href{https://en.wikipedia.org/wiki/Wu_wei}{Wu Wei (effortless action)}. Go with the flow, as in do not disregard the flow of the universe. Allow actions to unfold naturally. Let go of the desire to control everything. Let go and find peace.
\item
  \href{https://en.wikipedia.org/wiki/Hindu_philosophy}{\textbf{Hinduism}:} Concept of \href{https://en.wikipedia.org/wiki/Nishkama_Karma}{Nishkama Karma (detached action)}. Perform your duties without being attached to the fruits of your actions. So, focus on the present moment without being overly concerned with the future.
\item
  \href{https://en.wikipedia.org/wiki/Existentialism}{\textbf{Existentialism:}} Embrace the uncertainty of the future and the unknown, rather than relying on fixed beliefs or predetermined destinies. Enjoy the freedom!
\item
  \href{https://en.wikipedia.org/wiki/Mindfulness}{\textbf{Modern mindfulness}}: Live and stay in the present. Observe the thoughts and feelings without judgment.
\end{itemize}

\hypertarget{about-ikigai-and-having-a-sense-of-purpose}{%
\subsection{About ikigai and having a sense of purpose}\label{about-ikigai-and-having-a-sense-of-purpose}}

\textbf{My take:} Having a sense of purpose is important to be content in life. However, this purpose doesn't have to be grandiose. Purposelessness can be a purpose itself. As in, the purpose is to be existing or to be happy. That can be the main internal belief and can work fine, as long as you live in accordance that belief, and satisfy it.

The issue with happiness being the core belief or purpose is that: Happiness is an internal state, an emotion, thus it is unrealistic to expect to be happy all the time. Keeping this expectation at a reasonable level can be important. Besides that, I don't see any issue with existence being the sole purpose.

\textbf{Practical framework for ikigai:} The intersection of four elements: (1) What you love, (2) what you are good at, (3) what the world needs, and (4) what you can be paid for.

While the above framework is helpful in practice (since money and jobs are necessary in life under normal circumstances), in general, I do not believe this purpose have to be something specific, constant or explicitly defined. Spontaneous living and tackling a variety of interests can be perfectly fine too.

The key is to have something to look forward to.

  \bibliography{book.bib,packages.bib}

\end{document}
